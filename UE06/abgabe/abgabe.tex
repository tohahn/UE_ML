\documentclass[a4paper]{article}
\usepackage{hyperref}
%% Language and font encodings
\usepackage[english]{babel}

\usepackage[T1]{fontenc}


\usepackage[utf8]{inputenc}
\usepackage{amsmath}

\usepackage{caption}
\usepackage{graphicx}
\usepackage{tikz}
\usepackage{listings}
\usepackage{pgfplots}
\usepackage{placeins}

 
\graphicspath{ {results/} }
\usepackage{forest}


\tikzset{
  treenode/.style = {shape=rectangle, rounded corners,
                     draw, align=center,
                     top color=white, bottom color=blue!20},
  root/.style     = {treenode, font=\Large, bottom color=red!30},
  env/.style      = {treenode, font=\ttfamily\normalsize},
  dummy/.style    = {circle,draw}
}

\lstset{
	basicstyle=\footnotesize,
	tabsize=3,
	title=\lstname,
	breaklines=true
}

\addtolength{\oddsidemargin}{-.875in}
\addtolength{\evensidemargin}{-.875in}
\addtolength{\textwidth}{1.75in}

\addtolength{\topmargin}{-.875in}
\addtolength{\textheight}{1.75in}

\title{Machine Learning Übungsblatt 4}
\author{Ramon Leiser\and Tobias Hahn}

\begin{document}
\maketitle
\newpage
 \section{Begriffsdefinitionen}
 
\subsection{VC-Dimension eines Intervalls auf dem Zahlenstrahl}
Intervalle auf dem reellen Zahlenstrahl sind definierbar durch zwei Punkte, dem Anfangs und dem Endpunkt. Somit muss ein Klassifikator in der Lage sein diese Menge und alle ihre Untermengen zu zerschmettern. Die VC-Dimension beträgt also 2.

\subsection{Realisierbarkeitsannahme}
Googelt man "Realisierbarkeitannahme" bekommt man zwei Ergebnisse:
\begin{itemize}
\item Einen Foliensatz zur Wirtschaftsinformatik, der nichts mit Pac-lernen zu tun hat.
\item Die Angabe zu dieser Übung auf \texttt{www.tzi.de}
\end{itemize}
Auch unsere weitere Suche z.B. im four-germans-paper, sowie eine Wiederholung der Folien zur Pac-Lernbarkeit lieferten keine weitere Antworten.\\


\subsection{Vervollständigen sie den Satz ...}
\textit{Eine Hypothesenklasse $\mathcal{H}$  ist PAC-lernbar, falls es ein Lernverfahren $L$ und eine Funktion $m_H$ abhängig von $\epsilon$ und $\delta$ gibt, so dass für alle Verteilungen $D$ über $\mathcal{X}$ und jede Klassifizierungsfunktion $f : \mathcal{X} \rightarrow \{0, 1\}$ unter der Realisierbarkeitsannahme und $ m \geq m_H(\epsilon, \delta) $ unter $D$ identisch und unabhängig verteilten und mit $f$ beschrifteten Beispielen gilt, }\\
dass die VC-Dimension von D endlich ist.


\subsection{Fundamentalsatz der Lerntheorie}
Der Fundermentalsatz der Lerntheorie beschreibt den Zusammenhang zwischen PAC-Lernbarkeit und der VC-Dimension. Nach Paul Fischer genügt es einen Konsistenten Hypothesen-Finder zu haben, welcher Stichproben aus $C$ auf $\mathcal{H}$ abbildet, so dass die Hypothesen konsistent sind mit beiden Konzepten und zu zeigen, dass die VC-Dimension von $\mathcal{H}$ endlich ist um zu wissen dass nach dem PAC-Modell $\textnormal{VcDim}(\mathcal{H})/\epsilon$ Stichproben genügen um die PAC-Bedingung zu erfüllen. \footnote{Paul Fischer \textit{Algorithmisches Lernen} 1999 unter: https://books.google.de/books?id=34HzBQAAQBAJ\&pg=PA37\&lpg=PA37
}

\subsection{No-Free Lunch}
No-Free Lunch geht ursprünglich auf den Autor \textbf{ Robert A. Heinlein} und den Science Fiction Roman \textit{The Moon Is a Harsh Mistress } zurück und Entwickelte sich zu einer Bezeichnung in der Mathematik und Informatik für einen gewissen Zusammenhang bei Optimierungsproblemen und somit auch im Machine-Learning. \\
Das \textbf{No-free-Lunch-Theorem} besagt dass beim abstrahieren von Datensätzen kein einzelner Lösungsansatz auf der menge aller möglichen Probleme besser abschneidet als ein anderer. Jeder Lösungsansatz wäre bei einem Test auf alle möglichen Probleme nur genau so gut der Lösungsansatz alle Ergebnisse zu Raten.\\
Da die meisten Probleme in der Realität z.B. den Naturgesetzen genügen müssen sich Lösungsansätze nie auf allen möglichen Problemen beweisen.

\subsection{Sokoban}
Bei Sokoban verschiebt man Kästen in einem rasterisierten  Labyrinth. Man kann Kästen nur vor der Spielfigur her schieben und kann nur eine Kiste gleichzeitig bewegen. Es gibt eine Anzahl von vorgegebenen Zielfeldern auf die Jeweils ein beliebiger Kasten gestellt werden muss um ein Level ab zu schließen. Die Anzahl der gebrauchten Züge ergeben die Punktzahl für das Level (desto kleiner desto besser). \\
Dadurch dass man die Kästen nur schieben und nicht ziehen kann, ergeben sich schnell Zustände aus denen man die Kästen nicht mehr heraus bekommt, z.B. wen ein Kasten in einer Ecke im Labyrinth steht.

\subsection{PCA auf Iris Datensatz}
Wir haben versucht Korrelation im Iris Datensatz auf zu decken um so eventuell die Dimensisionalität des Problems zu senken. Der Iris-Datensatz besteht aus Daten zu drei Schwertlilienarten, welche anhand der aufgeführten Merkmale fast unterscheidbar sind. Zu jeder art gibt es 50 Einträge also insgesamt 150 Datenzeilen. Die Merkmale lauten:
\begin{enumerate}
\item Kelchblütenblatt-Länge.
\item Kelchblütenblatt-Breite.
\item Blütenblatt-Länge.
\item Blütenblatt-Breite.
\end{enumerate}
Die PCA in Weka zeigte die folgenden Eigenwerte mit zugehörigem Anteil an der gesamten Varianz an:
\begin{lstlisting}
eigenvalue	proportion	cumulative
  2.91082	  0.7277 	  0.7277 	-0.581petL-0.566petW-0.522sepL+0.263sepW
  0.92122	  0.23031	  0.95801	0.926sepW+0.372sepL+0.065petW+0.021petL
  0.14735	  0.03684	  0.99485	-0.721sepL+0.634petW+0.242sepW+0.141petL
\end{lstlisting}
Der letzte Eigenwert (und somit auch der zugehörige Eigenvektor) tragen nur $ 3\% $ zur Varianz der Daten bei und könnte deshalb außen vor gelassen werden.  \\
Besser mehr Information enthalten die Eigenvektoren sind die Eigenvektoren:
\begin{lstlisting}
Eigenvectors
 V1	 	 V2	 	 V3	
-0.5224	 0.3723	-0.721 	sepL
 0.2634	 0.9256	 0.242 	sepW
-0.5813	 0.0211	 0.1409	petL
-0.5656	 0.0654	 0.6338	petW
\end{lstlisting}
Der erste Eigenvektor wird relativ stark von allen Variablen beeinflusst. Man könnte ihn als die Pflanzengröße in Relation zur Kelchblüten-Breite betiteln, da die Kelchblüten-Breite in einem inversen Verhältnis zur Kelchblüten-Länge, sowie zu den anderen Variablen steht. \\
Der zweite Eigenvektor wird vor allem von den maßen der Kelchblüte bestimmt. Hier stehen die beiden Werte jedoch in positiver Korrelation. Man könnte diesen Eigenvektor als die Kelchblütengröße Bezeichnen.\\
Der dritte Eigenvektor beschreibt vor allem einen negativen Zusammenhang zwischen der Kelchblütenblatt-Breite und der Blütenblatt-Länge und ist dem ersten Eigenvektor ähnlich da er auch genau ein andersartiges Vorzeichen besitzt. Er trägt allerdings nur noch zu einem sehr kleinen Anteil der Varianz bei.\\
Der Letzte Eigenvektor kann nun bei der Basistransformation weggelassen werden, was die Dimensionalität des Datensatzes von vier auf drei Dimensionen senkt.\\
Trainiert man nun einen Naive-Bayes Klassifikator auf dem normalen und dem reduzierten Datensatz und vergleicht die Ergebnisse stellt man fest, dass der Reduzierte Datensatz wie zu erwarten nur leicht schlechter ist als die Klassifikation auf dem ganzen Datensatz. Zum Test wurde ein Cross-validation Verfahren benutzt mit 10 Faltungen. Hier ein Auszug der Ergebnisse:
\begin{itemize}
\item Voller Datensatz
\begin{lstlisting}
=== Confusion Matrix ===

  a  b  c   <-- classified as
 50  0  0 |  a = Iris-setosa
  0 48  2 |  b = Iris-versicolor
  0  4 46 |  c = Iris-virginica
  
Mean absolute error                      0.0342
\end{lstlisting}
\item Reduzierter Datensatz
\begin{lstlisting}
=== Confusion Matrix ===

  a  b  c   <-- classified as
 50  0  0 |  a = Iris-setosa
  0 44  6 |  b = Iris-versicolor
  0  5 45 |  c = Iris-virginica
  
Mean absolute error                      0.0717
\end{lstlisting}
\end{itemize}
Der Mittlere absolute Fehler steigt also von $3\%$ auf $7\%$. Daran erkennt man dass im Iris-Datensatz kaum redundante Korrelation der Daten steckt.
\section{Apriori Algorithmus}
\subsection{Entdeckung von Assoziazionsregeln}
Wir sind fälschlicherweise davon ausgegangen, dass der Algorithmus implementiert werden soll. Die folgende Berechnung ist der Output einer selbst geschriebenen Implementation des Algorithmus,welche mit abgegeben wird. Da das Hauptaugenmerk dieser Aufgabe nicht die Implementation ist wurde der Quellcode nur minimal dokumentiert. Die Ausgabe wurde aber mit einem Resultat einer Implementierung im Weka-Toolkit verglichen und ist korrekt.
\begin{lstlisting}
Transactions :
Transaction size: 1[Wasser]
Transaction size: 2[Brot, Bier]
Transaction size: 2[Cola, Bier]
Transaction size: 2[Wasser, Saft]
Transaction size: 3[Schokolade, Chips, Cola]
Transaction size: 3[Saft, Cola, Wein]
Transaction size: 3[Saft, Cola, Bier]
Transaction size: 3[Schokolade, Schinken, Brot]
Transaction size: 4[Saft, Cola, Bier, Wein]


\end{lstlisting}
Der erste Teil des Algorithmus besteht darin große, mehrmals vorkommende Itemsets in der Eingabe zu finden.
\begin{lstlisting}
-------------------------------------
|   Apriori 	 Part 	 	 One  	|
|   Generate Large Item Sets		|
-------------------------------------

---------------- k - Equals : 0 ----------------
[ 		  aprioriGen input :[Schokolade][Wasser][Saft][Brot][Bier][Cola][Wein]]
[ 		 after aprioriGen step join :[Bier, Cola][Wasser, Saft][Schokolade, Cola][Schokolade, Bier][Wasser, Brot][Schokolade, Wein][Schokolade, Brot][Wasser, Bier][Cola, Wein][Schokolade, Wasser][Brot, Wein][Brot, Cola][Saft, Cola][Schokolade, Saft][Saft, Brot][Saft, Bier][Wasser, Wein][Brot, Bier][Wasser, Cola][Saft, Wein][Bier, Wein]]
[ 		after aprioriGen step prune :[Bier, Cola][Wasser, Saft][Schokolade, Cola][Schokolade, Bier][Wasser, Brot][Schokolade, Wein][Schokolade, Brot][Wasser, Bier][Cola, Wein][Schokolade, Wasser][Brot, Wein][Brot, Cola][Saft, Cola][Schokolade, Saft][Saft, Brot][Saft, Bier][Wasser, Wein][Brot, Bier][Wasser, Cola][Saft, Wein][Bier, Wein]]



\end{lstlisting}
Hier wurden die neuen Hypothesen durch die Apriori-Gen Methode erzeugt. Es handelt sich im ersten Schritt um zweielementige Mengen. 
\begin{lstlisting}
new apriori-gen:[Bier, Cola][Wasser, Saft][Schokolade, Cola][Schokolade, Bier][Wasser, Brot][Schokolade, Wein][Schokolade, Brot][Wasser, Bier][Cola, Wein][Schokolade, Wasser][Brot, Wein][Brot, Cola][Saft, Cola][Schokolade, Saft][Saft, Brot][Saft, Bier][Wasser, Wein][Brot, Bier][Wasser, Cola][Saft, Wein][Bier, Wein]
Support[Bier, Cola] 		 is 0.33333334
Support[Wasser, Saft] 		 is 0.11111111
Support[Schokolade, Cola] 		 is 0.11111111
Support[Schokolade, Bier] 		 is 0.0
Support[Wasser, Brot] 		 is 0.0
Support[Schokolade, Wein] 		 is 0.0
Support[Schokolade, Brot] 		 is 0.11111111
Support[Wasser, Bier] 		 is 0.0
Support[Cola, Wein] 		 is 0.22222222
Support[Schokolade, Wasser] 		 is 0.0
Support[Brot, Wein] 		 is 0.0
Support[Brot, Cola] 		 is 0.0
Support[Saft, Cola] 		 is 0.33333334
Support[Schokolade, Saft] 		 is 0.0
Support[Saft, Brot] 		 is 0.0
Support[Saft, Bier] 		 is 0.22222222
Support[Wasser, Wein] 		 is 0.0
Support[Brot, Bier] 		 is 0.11111111
Support[Wasser, Cola] 		 is 0.0
Support[Saft, Wein] 		 is 0.22222222
Support[Bier, Wein] 		 is 0.11111111
survivors: [Bier, Cola][Saft, Cola][Saft, Bier][Saft, Wein][Cola, Wein]



\end{lstlisting}
Alle Mengen, die nicht dem minimalen Support entsprechen wurden entfernt. Basierend auf den noch vorhandenen Mengen werden nun dreielementige Mengen mit Apriori-Gen erzeugt.
\begin{lstlisting}
---------------- k - Equals : 1 ----------------
[ 		  aprioriGen input :[Bier, Cola][Saft, Cola][Saft, Bier][Saft, Wein][Cola, Wein]]
[ 		 after aprioriGen step join :[Saft, Bier, Wein][Saft, Bier, Cola][Bier, Cola, Wein][Saft, Cola, Wein]]
[ 		after aprioriGen step prune :[Saft, Bier, Wein][Saft, Bier, Cola][Bier, Cola, Wein][Saft, Cola, Wein]]

new apriori-gen:[Saft, Bier, Wein][Saft, Bier, Cola][Bier, Cola, Wein][Saft, Cola, Wein]
Support[Saft, Bier, Wein] 		 is 0.11111111
Support[Saft, Bier, Cola] 		 is 0.22222222
Support[Bier, Cola, Wein] 		 is 0.11111111
Support[Saft, Cola, Wein] 		 is 0.22222222
survivors: [Saft, Bier, Cola][Saft, Cola, Wein]

---------------- k - Equals : 2 ----------------
[ 		  aprioriGen input :[Saft, Bier, Cola][Saft, Cola, Wein]]
[ 		 after aprioriGen step join :[Saft, Bier, Cola, Wein]]
[ 		after aprioriGen step prune :[Saft, Bier, Cola, Wein]]

new apriori-gen:[Saft, Bier, Cola, Wein]
Support[Saft, Bier, Cola, Wein] 		 is 0.11111111
survivors: 
None! Iteration Over.!



\end{lstlisting}
Es sind nun keine Mengen mehr vorhanden, die noch den minimalen Support erfüllen. Größere Mengen hätten einen noch kleineren Support und werden deshalb nicht verfolgt.
\begin{lstlisting}
Large Itemsets:
<Item Set:[Bier, Cola] has support0.33333334
<Item Set:[Wasser] has support0.22222222
<Item Set:[Saft, Bier, Cola] has support0.22222222
<Item Set:[Saft] has support0.44444445
<Item Set:[Cola] has support0.5555556
<Item Set:[Saft, Cola, Wein] has support0.22222222
<Item Set:[Cola, Wein] has support0.22222222
<Item Set:[Schokolade] has support0.22222222
<Item Set:[Saft, Cola] has support0.33333334
<Item Set:[Brot] has support0.22222222
<Item Set:[Saft, Bier] has support0.22222222
<Item Set:[Bier] has support0.44444445
<Item Set:[Wein] has support0.22222222
<Item Set:[Saft, Wein] has support0.22222222


\end{lstlisting}
Im zweiten Teil des Algorithmus werden nun Regeln erzeugt. Zunächst wird den großen Itemsets jeweils ein Element entnommen, auf das von den restlichen Einträgen aus geschlossen werden soll. 
\begin{lstlisting}

-------------------------------------
|  Apriori 	 Part 		 Two	|
|  Generate  Association Rules		|
-------------------------------------
Confidence[Cola]-->[Bier] 		 is 0.59999996
Confidence[Bier]-->[Cola] 		 is 0.75
Confidence[Bier, Cola]-->[Saft] 		 is 0.6666666
Confidence[Saft, Cola]-->[Bier] 		 is 0.6666666
Confidence[Saft, Bier]-->[Cola] 		 is 1.0
Confidence[Cola, Wein]-->[Saft] 		 is 1.0
Confidence[Saft, Wein]-->[Cola] 		 is 1.0
Confidence[Saft, Cola]-->[Wein] 		 is 0.6666666
Confidence[Wein]-->[Cola] 		 is 1.0
Confidence[Cola]-->[Wein] 		 is 0.39999998
Confidence[Cola]-->[Saft] 		 is 0.59999996
Confidence[Saft]-->[Cola] 		 is 0.75
Confidence[Bier]-->[Saft] 		 is 0.5
Confidence[Saft]-->[Bier] 		 is 0.5
Confidence[Wein]-->[Saft] 		 is 1.0
Confidence[Saft]-->[Wein] 		 is 0.5


\end{lstlisting}
Alle Hypothesen unter der minimalen Confidence werden entfernt. Es bleiben lediglich Hypothesen die von den folgenden Mengen ausgehen. Da all diese Hypothesen auf \texttt{Cola} oder \texttt{Saft} Schließen werden diese beiden Element für den Apriori-Gen verwendet.
\begin{lstlisting}
first hypotheses: [Saft][Saft, Bier][Bier][Wein][Saft, Wein][Cola, Wein]

[ 		  aprioriGen input :[Saft][Cola]]
[ 		 after aprioriGen step join :[Saft, Cola]]
[ 		after aprioriGen step prune :[Saft, Cola]]

new apriori-gen:[Saft, Cola]
Confidence[Bier]-->[Saft, Cola] 		 is 0.5
0.5
Confidence[Wein]-->[Saft, Cola] 		 is 1.0
1.0
survivors: 
None! You Done!



\end{lstlisting}
Damit ist der Algorithmus beendet und es folgen die Ergebnisse. 
\begin{lstlisting}

Saft	--->	[[Cola]]
Saft	Bier	--->	[[Cola]]
Bier	--->	[[Cola]]
Wein	--->	[[Saft], [Saft, Cola], [Cola]]
Saft	Wein	--->	[[Cola]]
Cola	Wein	--->	[[Saft]]

\end{lstlisting}
\subsection{Verbesserung des grundlegenden Algorithmus}
Eine Variante vom Apriori-Algorithmus ist der Max-Miner. Er Arbeitet ähnlich wie der Apriori, generiert jedoch nicht alle häufigen Teilmengen aus den Transaktionen sondern direkt die maximal großen. Er arbeite auf einer Baumstruktur. \\
Außerdem fanden wir den M-Apriori Algorithmus, welcher beim generieren der Itemsets speichert welches der Items im Set den kleinsten Support hatte. Außerdem Besteht zu jedem Item eine Referenz in welchen der Transaktionen es auftaucht. Nun kann zum bestimmen des Supports darauf zurückgegriffen werden und nur diese Transaktionen nach dem Itemset durchsucht werden.\\
Durch ihre Verbesserungen sind die beiden Algorithmen schneller als der normale Apriori-Algorithmus ohne sein Ergebniss zu verändern, nur durch das einsparen von unnötigen Lookups und Erzeugung von Sets die dem minimalen Support nicht standhalten.
\end{document}

